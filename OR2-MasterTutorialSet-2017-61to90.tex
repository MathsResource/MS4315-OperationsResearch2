 \documentclass[a4paper,12pt]{article}
 %%%%%%%%%%%%%%%%%%%%%%%%%%%%%%%%%%%%%%%%%%%%%%%%%%%%%%%%%%%%%%%%%%%%%%%%%%%%%%%%%%%%%%%%%%%%%%%%%%%%%%%%%%%%%%%%%%%%%%%%%%%%%%%%%%%%%%%%%%%%%%%%%%%%%%%%%%%%%%%%%%%%%%%%%%%%%%%%%%%%%%%%%%%%%%%%%%%%%%%%%%%%%%%%%%%%%%%%%%%%%%%%%%%%%%%%%%%%%%%%%%%%%%%%%%%%
 \usepackage{eurosym}
 \usepackage{vmargin}
 \usepackage{amsmath}
 \usepackage{multicol}
 \usepackage{graphics}
 \usepackage{enumerate}
 \usepackage{epsfig}
 \usepackage{framed}
 \usepackage{subfigure}
 \usepackage{fancyhdr}
 
 \setcounter{MaxMatrixCols}{10}
 %TCIDATA{OutputFilter=LATEX.DLL}
 %TCIDATA{Version=5.00.0.2570}
 %TCIDATA{<META NAME="SaveForMode" CONTENT="1">}
 %TCIDATA{LastRevised=Wednesday, February 23, 2011 13:24:34}
 %TCIDATA{<META NAME="GraphicsSave" CONTENT="32">}
 %TCIDATA{Language=American English}
 
 %\pagestyle{fancy}
 \setmarginsrb{20mm}{0mm}{20mm}{25mm}{12mm}{11mm}{0mm}{11mm}
 %\lhead{MA4413 2013} \rhead{Mr. Kevin O'Brien}
 %\chead{Midterm Assessment 1 }
 %\input{tcilatex}
 
 \begin{document}
 \noindent (\textbf{Remark} : Question 61 to 70 are not examinable in End of Semester Exam. Questions 73 to 76 are indicative of potential exam questions.)
 	\begin{enumerate}
 		\setcounter{enumi}{60}	
 		%========================================%
 		
 		% grid 05 % level 0
 		
 		
 		\item Use LP software to solve the following IP problem, where $x,y$ are integers that are $\geq 0$.
 		\begin{framed}
 			\begin{verbatim}
 			maximize p = 5x + 7y subject to
 			30x + 33y <= 2100
 			50x + 24y <= 2400
 			x <=45
 			y <=60
 			
 			\end{verbatim}
 		\end{framed}
 		
 		
 		%========================================%
 		
 		% grid 02 % level 1
 		
 		
 		
 		\item Use LP software to solve the following IP problem, where $x,y$ are integers that are $\geq 0$.
 		\begin{framed}
 			\begin{verbatim}
 			maximize p = 3x + y subject to
 			5x - 6y <= 90
 			2x + 5y <= 190
 			y <= 32
 			X <= 38
 			\end{verbatim}
 		\end{framed}
 		
 		
 		%========================================%
 		
 		% grid 09 % level 1
 		
 		\item Use LP software to solve the following IP problem, where $x,y$ are integers that are $\geq 0$.
 		\begin{framed}
 			\begin{verbatim}
 			maximize p = 3x + y subject to
 			2x - y <= 40
 			2x + 3y <= 64
 			y <= 15
 			x <= 22
 			\end{verbatim}
 		\end{framed}
 		
 		%========================================%
 		
 		% grid 10 % level 1
 		
 		\item Use LP software to solve the following IP problem, where $x,y$ are integers that are $\geq 0$.
 		\begin{framed}
 			\begin{verbatim}
 			maximize p = 7x + 3y subject to
 			2y<=63
 			2x + 6y <= 205
 			2x + 2y <= 109
 			4x + y <= 180
 			x <= 43
 			\end{verbatim}
 		\end{framed}
 		%========================================%
 		\newpage	
 		% grid 11 % level 1
 		\item Use LP software to solve the following IP problem, where $x,y$ are integers that are $\geq 0$.
 		\begin{framed}
 			\begin{verbatim}
 			maximize p = 7x + 5y subject to
 			7x + 9y <=  450
 			6x + 11y <= 380
 			y <= 35
 			x <= 50
 			\end{verbatim}
 		\end{framed}
 		%========================================%
 		
 		% grid 01 % level 2
 		\item Use LP software to solve the following IP problem, where $x,y$ are integers that are $\geq 0$.
 		\begin{framed}
 			\begin{verbatim}
 			maximize p = 3x + y subject to
 			7x - 3y <= 70
 			2x + 5y <= 180
 			y <= 30
 			\end{verbatim}
 		\end{framed}
 		%========================================%
 		
 		% grid 13 % level 2
 		\item Use LP software to solve the following IP problem, where $x,y$ are integers that are $\geq 0$.
 		\begin{framed}
 			\begin{verbatim}
 			maximize p = 5x + 2y subject to
 			30x + 33y <= 2100
 			50x + 24y <= 2400
 			x <=45
 			y <=60
 			\end{verbatim}
 		\end{framed}
 		%========================================%
 		
 		% grid 14 % level 2
 		\item Use LP software to solve the following IP problem, where $x,y$ are integers that are $\geq 0$.
 		\begin{framed}
 			\begin{verbatim}
 			maximize p = 3x + 7y subject to
 			2x - y <= 40
 			2x + 3y <= 90
 			y <= 25
 			\end{verbatim}
 		\end{framed}
 		%========================================%
 		\newpage	
 		% grid 15 % level 2
 		\item Use LP software to solve the following IP problem, where $x,y$ are integers that are $\geq 0$.
 		\begin{framed}
 			\begin{verbatim}
 			maximize p = 3x + 8y subject to
 			2x - y <= 40
 			2x + 3y <= 90
 			y <= 25
 			\end{verbatim}
 		\end{framed}
 		%========================================%
 		
 		% grid 07 % 4 Levels
 		
 		\item Use LP software to solve the following IP problem, where $x,y$ are integers that are $\geq 0$.
 		\begin{framed}
 			\begin{verbatim}
 			maximize p = 7x + 5y subject to
 			2y<=63
 			2x + 6y <= 205
 			2x + 2y <= 109
 			4x + y <= 180
 			x <= 43
 			\end{verbatim}
 		\end{framed}
  %------------------------------------------------------------%
  
  
  
  
  \item OBKB Investments is considering investments into 5 projects: A, B, C, D, and E.
  
  Each project has an initial cost, an expected profit rate (one year from now) expressed as a
  percentage of the initial cost, and an associated risk of failure.
  These numbers are given in the table below:
  
  \begin{center}
  	\begin{tabular}{|c|c|c|c|c|c|}
  		\hline  & A & B & C & D & E \\ 
  		\hline Initial Cost & 1.5 & 0.9 & 0.7 & 1.5 & 2.1 \\ 
  		\hline Profit Rate & 10\% & 15\% & 10\% & 12\% & 10\% \\ 
  		\hline Failure Risk & 6\% & 4\% & 6\% & 5\% & 4\%  \\ 
  		\hline 
  	\end{tabular} 
  \end{center}
  \begin{enumerate}\item  Provide a formulation to choose the projects that maximize total
  	expected profit, such that OBKB Investments does not invest more than
  	5M dollars and its average failure risk is not over 5\%.
  	
  	For example, if  OBKB Investments invests only into A and B, it invests
  	only 2.5M dollars and its average failure risk is $(6\%+4\%)/2=5\%$.%\marks{3\%}
  	
  	\item Suppose that if A is chosen, B must be chosen. Modify your
  	formulation.%\marks{3\%}
  	\item Suppose that if C is chosen, E must not be chosen. Modify your
  	formulation.%\marks{3\%}
  	\item   Suppose that if A and C are chosen, D must be chosen. Modify your
  	formulation.%\marks{3\%}
  \end{enumerate}
%======================================%
\newpage
   		
\item 	ERBIZAKIP Investments is considering investments into 6 projects: A, B, C, D, E and F.
  
  Each project has an initial cost, an expected profit rate (one year from now) expressed as a
  percentage of the initial cost, and an associated risk of failure.
  These numbers are given in the table below:
  
  \begin{center}
  	\begin{tabular}{|c||c|c|c|c|c|c|}
  		\hline  & A & B & C & D & E & F \\ 
  		\hline Initial Cost & 1.8 & 1.5 & 1.1 & 1.8 & 2.1 & 3.2 \\ 
  		\hline Profit Rate & 11\% & 13\% & 10\% & 12\% & 11\% & 9\% \\ 
  		\hline Failure Risk & 6\% & 4\% & 5.5\% & 5\% & 4.5\%  & 4.5\%\\ 
  		\hline 
  	\end{tabular} 
  \end{center}
  \begin{enumerate}[(i)] \item  Provide a formulation to choose the projects that maximize total
  	expected profit, such that ERBIZAKIP Investments does not invest more than
  	5M dollars and its average failure risk is not over 5\%. 
  	
  	You may assuming equal weighting for each project when determining average risk. For example, if ERBIZAKIP Investments invests only into A,B and C, it invests
  	only 4.4M dollars and its average failure risk is $(6\%+4\% +5\%)/3=5\%$.
  	
  	\item Suppose that if C is chosen, D must be chosen. Modify your
  	formulation.
  	%\item Suppose that if E is chosen, F must not be chosen. Modify your
  	%	formulation.\marks{2\%}
  	\item   Suppose that if A and C are chosen, D must be chosen. Modify your
  	formulation.
  	\item Suppose that only two projects, at most, can be chosed from A, B and C.  Modify your
  	formulation.
  \end{enumerate}		
  
 \item Use Simplex Tableaux to solve the follow maximization problem. (Assume x and y $>$ 0).
 \begin{verbatim}  
 maximize p = 3x + 2y subject to
 2x - 1y <= 8
 x + 2y <= 14
 \end{verbatim}
 
 % Vertex Lines Through Vertex   Value of Objective
 % (6,4)  2x-1y = 8; x+2y = 14   26 Maximum
 % (4,0)  2x-1y = 8; y = 0       12
 % (0,7)  x+2y = 14; x = 0       14
 % (0,0)  x = 0; y = 0           0
 
 
 %===================================%
 \item Use Simplex Tableaux to solve the follow maximization problem. (Assume x and y $>$ 0).
 \begin{verbatim} 
 maximize p = 3x + 2y subject to
 2x - 1y <= 9
 x + 2y <= 17
 \end{verbatim}
 % Vertex   Lines Through Vertex   Value of Objective
 % (7,5)    2x-1y = 9; x+2y = 17   31 Maximum
 % (4.5,0)  2x-1y = 9; y = 0       13.5
 % (0,8.5)  x+2y = 17; x = 0       17
 % (0,0)    x = 0; y = 0           0
 
 %====================================%
 \item Use Simplex Tableaux to solve the follow maximization problem. (Assume x and y $>$ 0).
 \begin{verbatim}


 \end{verbatim}
 % Vertex   Lines Through Vertex    Value of Objective
 % (9,5)    2x-1y = 13; x+2y = 19   37 Maximum
 % (6.5,0)  2x-1y = 13; y = 0       19.5
 % (0,9.5)  x+2y = 19; x = 0        19
 % (0,0)    x = 0; y = 0            0
 
%======================================%
\item Use Simplex Tableaux to solve the follow maximization problem. (Assume x and y $>$ 0).
\begin{verbatim}
maximize p = 3x + 2y subject to
2x - 1y <= 13
x + 2y <= 19
\end{verbatim}
%====================================%
% Quesion 77 		  
\item Essay style question on the construction and interpretation of ROC curves.
\begin{itemize}
	\item What is the purpose of the plot?
	\item What are the Axes?
	\item Show how to interpret the curve using a few sketches.
\end{itemize}
%====================================%
% Quesion 78
\item Encoding constraints for Binary IP problems.
\begin{itemize}
	\item Similar to Question 72.
\end{itemize}
\item 

A normally distributed quality characteristic is monitored through the use of control charts. These charts have the following parameters. All charts are in control.
\begin{center}
	\begin{tabular}{|c|c|c|c|}
		\hline  & LCL & Centre Line & UCL \\
		\hline $\bar{X}$-Chart & 542 & 550 & 558 \\
		\hline $R$-Chart & 0 & 8.236 & 16.504 \\ \hline
	\end{tabular}
\end{center}

\begin{itemize}
	\item[(i.)] (2 marks) What sample size is being used for this analysis?
	\item[(ii.)](2 marks) Estimate the mean of the standard deviations $\bar{s}$ for this process.
	\item[(iii.)] (2 marks) Compute the control limits for the process standard deviation chart (i.e. the s-chart).
\end{itemize}

\item A normally distributed quality characteristic is monitored through the use of control charts. These charts have the following parameters. All charts are in control.
\begin{center}
	\begin{tabular}{|c|c|c|c|}
		\hline  & LCL & Centre Line & UCL \\
		\hline $\bar{X}$-Chart & 1995 & 2000 & 2005 \\
		\hline $R$-Chart & 0 & 21 & 44.394 \\ \hline
	\end{tabular}
\end{center}

\begin{itemize}
	\item[(i.)] (2 Marks) What sample size is being used for this analysis?
	\item[(ii.)] (2 Marks) Estimate the mean of the process standard deviations $\bar{s}$.
	\item[(iii.)] (2 Marks) Compute the control limits for the process standard deviation chart (i.e. the s-chart).
\end{itemize}


\item 
The \textbf{Nelson Rules} are a set of eight decision rules for detecting ``out-of-control" or non-random conditions on control charts. These rules are applied to a control chart on which the magnitude of some variable is plotted against time. The rules are based on the mean value and the standard deviation of the samples.\\

\begin{itemize}
	\item[(i)] Discuss any four of these rules, and how they would be used to detect ``out of control" processes. Support your answer with sketch.
\end{itemize}

\bigskip 
\begin{framed}
	\noindent \textit{In your answer, you may make reference to the following properties of the Normal Distribution. Consider the random variable $X$ distributed as
		\[X \sim \mathcal{N}(\mu,\sigma^2)\]
		where $\mu$ is the mean and $\sigma^2$ is the variance of an random variable $X$.}
	\begin{itemize}
		\item $\Pr( \mu - 1\sigma \leq X \leq \mu + 1\sigma ) = 0.6827$
		\item $\Pr( \mu - 2\sigma \leq X \leq \mu + 2\sigma ) = 0.9545$
		\item $\Pr( \mu - 3\sigma \leq X \leq \mu + 3\sigma )= 0.9973$
		
	\end{itemize}
\end{framed}
\item  (5 Marks) By removing all strategies which are dominated by strict pure or mixed strategies, derive the reduced version of the following 2-player zero-sum matrix game:

\begin{center}
	
	\begin{tabular}{|c||c|c|c|}
		\hline
		& D         &E       &F    \\
		\hline \hline
		A & (1,1)& (6,4)& (6,9) \\ \hline 
		B & (2,6)& (0,8)& (4,7) \\ \hline 
		C & (3,5)& (1,2)& (5,3) \\ \hline 
		
		
		\hline
		
	\end{tabular}
\end{center} % \marks{8}
 Derive the minimax strategies and value of the above game.% \marks{9}

 	\end{enumerate}
 \end{document}

\end{enumerate} 
\end{document}